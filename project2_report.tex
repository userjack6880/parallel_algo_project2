\documentclass{article}
\usepackage{parskip}
\usepackage{amsmath}
\usepackage{graphicx}
\graphicspath{ {./} }
\usepackage{tikz}
\usetikzlibrary{shapes.geometric, arrows, plotmarks}
\tikzstyle{startstop} = [rectangle, rounded corners, minimum width=3cm, minimum height=1cm, text centered, draw=black]
\tikzstyle{process} = [rectangle, minimum width=3cm, minimum height=1cm, text centered, draw=black]
\tikzstyle{decision} = [diamond, aspect=2, minimum width=3cm, minimum height=1cm, text centered, draw=black]
\tikzstyle{arrow} = [thick,->,>=stealth]
\usepackage{float}

\setlength{\belowcaptionskip}{10pt}

\title{Project 2: Parallel Fluid Dynamics Simulation}
\author{John Bradley}
\date{\today}

\begin{document}
  \maketitle

  \section{Introduction}

  This report aims to document the implementation and analysis of a program
  that utilizes OpenMP to implement a computational fluid dynamics model
  suitable for modeling behavior of turbulent flows. This report will cover the
  methods and techniques utilized for implementing the fluid dynamics model
  using OpenMP, analyze and estimate the expected performance of the 
  implementation, present the results from experimentation, compare and contrast 
  the expected performance to the observed results, and conclude with insights 
  and possible changes that could be made to the implementation
  
  \section{Methods and Techniques}

  

  \section{Analysis}

  

  \section{Results}

 

  \section{Synthesis}

 

  \section{Conclusion}



\end{document}

